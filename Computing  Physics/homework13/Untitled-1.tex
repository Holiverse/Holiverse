\documentclass[11pt,a4paper,titlepage,openany]{book}
\usepackage[space]{ctex}
\usepackage{mathrsfs,amssymb,amsfonts,amsmath,bm,ntheorem,graphicx}
\usepackage[paperwidth=185mm,paperheight=260mm,text={148mm,210mm},left=21mm,includehead,vmarginratio=1:1]{geometry}
\usepackage{fancyhdr,titlesec,enumerate}
\begin{document}
\pagestyle{fancy}
\fancyhf{}
\fancyhead[EL,OR]{\thepage}
\fancyhead[OC]{\nouppercase{\heiti\rightmark}}
\fancyhead[EC]{\nouppercase{\heiti\leftmark}}
\fancypagestyle{plain}{\renewcommand{\headrulewidth}{0pt}\fancyhf{}}
\theoremstyle{plain}
\newcounter{Proposition}[section]
\newenvironment{Proposition}[1][]{{\par\normalfont\bfseries ����~\stepcounter{Proposition}\arabic{Proposition}#1~~}\kaishu}{\par}
\newcounter{Corollary}[section]
\newenvironment{Corollary}[1][]{{\par\normalfont\bfseries ����~\stepcounter{Corollary}\arabic{Corollary}#1~~}\kaishu}{\par}
\newcounter{Theorem}[section]
\newenvironment{Theorem}[1][]{{\par\normalfont\bfseries ����~\stepcounter{Theorem}\arabic{Theorem}#1~~}\kaishu}{\par}
\newcounter{Lemma}[section]
\newenvironment{Lemma}[1][]{{\par\normalfont\bfseries ����~\stepcounter{Lemma}\arabic{Lemma}#1~~}\kaishu}{\par}
\newcounter{Property}[section]
\newenvironment{Property}[1][]{{\par\normalfont\bfseries ����~\stepcounter{Property}\arabic{Property}#1~~}\kaishu}{\par}
\newcounter{Assertion}[section]
\newenvironment{Assertion}[1][]{{\par\normalfont\bfseries ����~\stepcounter{Assertion}\arabic{Assertion}#1~~}\kaishu}{\par}
\newenvironment{Proof}{{\par{\heiti ֤��}~~}}{\hfill $\square$ \par\hfill\par}
\newcounter{Example}[section]
\newenvironment{Example}[1][]{{\par\normalfont\bfseries ��~\stepcounter{Example}\arabic{Example}#1~~}\songti}{\hfill\par\hfill\par}
\newcounter{Def}[section]
\newenvironment{Def}[1][]{{\par\normalfont\bfseries ����~\stepcounter{Def}\arabic{Def}#1~~\songti}}{\par}
\newcounter{Note}[section]
\newenvironment{Note}[1][]{{\par\normalfont\bfseries ע~\stepcounter{Note}\arabic{Note}#1~~}\songti}{\par}
\title{{\zihao{0}\heiti Homework10}}
\author{��ʥ��}
\maketitle
\chapter{Method Analysis}
\section{Count}
It's easy to randomly walk in different dimensions, just walking in one dimension and then multiply them individually. \par
The question is how to find the particle back to the zero, and we can use a temp-variable and count the number of the 0 at each dimension component after randomly walking. If the temp-variable equals the dimension, then return a true value and end the random walking, then the total counter will plus 1.
\chapter{Result}
We generate 10000 particles and set the maximal steps to 10000.\par
For d=1:\par
\includegraphics{d=1.pdf}\par
\includegraphics{d=1_fit.pdf}\par
As we can see in the graphic, for situation d=1, it's nearly a linear relation between $ln(P)\sim ln(N)$.\par
For d=2:\par
\includegraphics{d=2.pdf}\par
\includegraphics{d=2_fit.pdf}\par
The linear relation degenerate to a dependent relation.\par
For d=3:\par
\includegraphics{d=3.pdf}\par
\includegraphics{d=3_fit.pdf}\par
There is no dependent relation indeed.\par
So we cannot define a exponential degree for the probability, maybe comes from the fluctuation of the random walk.
\chapter{Others}
\section{appendix}
1.dat\par
2.dat\par
3.dat\par
d=1.pdf\par
d=1-fit.pdf\par
d=2.pdf\par
d=2-fit.pdf\par
d=3.pdf\par
d=3-fit.pdf\par
Untitled-1.tex\par
Untitled-1.cpp\par
CP13DASW.exe\par
\section{reference}
C++ primer,2003.11,Stanley B.Lippman;Josee Lajoie;Barbara E.Moo,Pearson Education Asia LTD.
\end{document} 