\documentclass[11pt,a4paper,titlepage,openany]{book}
\usepackage[space]{ctex}
\usepackage{mathrsfs,amssymb,amsfonts,amsmath,bm,ntheorem,graphicx}
\usepackage[paperwidth=185mm,paperheight=260mm,text={148mm,210mm},left=21mm,includehead,vmarginratio=1:1]{geometry}
\usepackage{fancyhdr,titlesec,enumerate}
\begin{document}
\pagestyle{fancy}
\fancyhf{}
\fancyhead[EL,OR]{\thepage}
\fancyhead[OC]{\nouppercase{\heiti\rightmark}}
\fancyhead[EC]{\nouppercase{\heiti\leftmark}}
\fancypagestyle{plain}{\renewcommand{\headrulewidth}{0pt}\fancyhf{}}
\theoremstyle{plain}
\newcounter{Proposition}[section]
\newenvironment{Proposition}[1][]{{\par\normalfont\bfseries ����~\stepcounter{Proposition}\arabic{Proposition}#1~~}\kaishu}{\par}
\newcounter{Corollary}[section]
\newenvironment{Corollary}[1][]{{\par\normalfont\bfseries ����~\stepcounter{Corollary}\arabic{Corollary}#1~~}\kaishu}{\par}
\newcounter{Theorem}[section]
\newenvironment{Theorem}[1][]{{\par\normalfont\bfseries ����~\stepcounter{Theorem}\arabic{Theorem}#1~~}\kaishu}{\par}
\newcounter{Lemma}[section]
\newenvironment{Lemma}[1][]{{\par\normalfont\bfseries ����~\stepcounter{Lemma}\arabic{Lemma}#1~~}\kaishu}{\par}
\newcounter{Property}[section]
\newenvironment{Property}[1][]{{\par\normalfont\bfseries ����~\stepcounter{Property}\arabic{Property}#1~~}\kaishu}{\par}
\newcounter{Assertion}[section]
\newenvironment{Assertion}[1][]{{\par\normalfont\bfseries ����~\stepcounter{Assertion}\arabic{Assertion}#1~~}\kaishu}{\par}
\newenvironment{Proof}{{\par{\heiti ֤��}~~}}{\hfill $\square$ \par\hfill\par}
\newcounter{Example}[section]
\newenvironment{Example}[1][]{{\par\normalfont\bfseries ��~\stepcounter{Example}\arabic{Example}#1~~}\songti}{\hfill\par\hfill\par}
\newcounter{Def}[section]
\newenvironment{Def}[1][]{{\par\normalfont\bfseries ����~\stepcounter{Def}\arabic{Def}#1~~\songti}}{\par}
\newcounter{Note}[section]
\newenvironment{Note}[1][]{{\par\normalfont\bfseries ע~\stepcounter{Note}\arabic{Note}#1~~}\songti}{\par}
\title{{\zihao{0}\heiti Phase space }}
\author{��ʥ��}
\maketitle

\newpage

\chapter{}
Both A and B are right in some point, but the question is what is the phase space and possibility density in $Ising$ model. First we should notice that it is a quantum system, and we can know from the Hamiltonian $H=\sum\limits_{i}\beta_{i}\sigma_{i}+\sum\limits_{<ij>}J_{ij}\sigma_{i}\sigma_{j}$, there is no relation with $x$ and $p$, so we can actually construct a symplectic structure by a pair coniancal variables $x$ and $p$ like A said, but useable. And we find the parameter define the Hamiltonian is $\sigma$, but it doesn't means that we have to set the spin $\sigma$ be the coniancal variables. Actually we can use wave functions in quantum mechanics to describe the different state, and of course the coefficient function is continus, and then we can get a good variable to construct the phase space. we can also use the field to construct the symplectic structure, which a more natural view point. By the coniancal quantization, we get the dual variable of the field.\par
As for $Liouville$ theorem, that is not a concept for the classical mechanics here, but a quantum system, which derived directly from the conservation of the local possibility.\par
The ensemble is a very general concept without the chosen of the coniancal variables. The main idea of ensemble is put a very large of same system together, and let them involution individually.\par
The Markov chain is used to describe the transition rate between different stable states, the state need to be stable, and we can write it as a matrix.\par
\end{document} 